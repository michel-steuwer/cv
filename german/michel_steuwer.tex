\documentclass[11pt,a4paper]{moderncv}

\renewcommand*{\familydefault}{\sfdefault}


\urlstyle{same}

\usepackage{xstring}
\def\FormatName#1{%
  \IfSubStr{#1}{Steuwer}{\textbf{#1}}{#1}%
}

\moderncvtheme[blue]{classic}
\usepackage[utf8]{inputenc}
% adjust the page margins
\usepackage[scale=0.8]{geometry}
\AtBeginDocument{\recomputelengths}

% personal data
\firstname{Michel}
\familyname{Steuwer}
\address{
  Informatics Forum -- 1.02\\
  10 Crichton Street\\
  Edinburgh EH8 9AB\\}{
  United Kingdom}
\email{michel.steuwer@ed.ac.uk}

% divide publications
\usepackage[style=numeric-comp,
            sorting=none, % keep order as in the bib file ...
            defernumbers,maxbibnames=50]{biblatex}

\defbibenvironment{bibliography}
  {\list
     {\printfield{year}\hspace{1em}\printtext[labelnumberwidth]{\printfield{prefixnumber}\printfield{labelnumber}}}
     {\setlength{\topsep}{0pt}% layout parameters from moderncvstyleclassic.sty
      \setlength{\labelwidth}{\hintscolumnwidth}%
      \setlength{\labelsep}{\separatorcolumnwidth}%
      \leftmargin\labelwidth%
      \advance\leftmargin\labelsep}%
      \sloppy\clubpenalty4000\widowpenalty4000}
  {\endlist}
  {\item}

% Highlight my Name in bold
\DeclareNameFormat{author}{%
\ifthenelse{\equal{#1}{Steuwer}}%
    {{\bfseries\ifblank{#4}{}{#4\space}#1}}%
    {\ifblank{#4}{}{#4\space}#1}%
\ifthenelse{\value{listcount}<\value{liststop}}%
    {\addcomma\space}
    {}}

% Only print a year once
\newcounter{currentYear}
\DeclareFieldFormat{year}{%
\ifthenelse{\equal{#1}{\arabic{currentYear}}}%
    {}
    {\setcounter{currentYear}{#1}{\bfseries #1}}}

\bibliography{michel_steuwer}

%----------------------------------------------------------------------------------
%            content
%----------------------------------------------------------------------------------
\begin{document}
\nocite{*} % cite everything ...
\maketitle
\vspace{-1em}
\begin{minipage}[t]{0.5\textwidth}
  \section{Angaben zur Person}
  \cvline{Geburtstag}{21. Mai 1985}
  \cvline{Geburtsort}{Duisburg, Deutschland}
  \cvline{Nationalität}{Deutsch}
\end{minipage}
\begin{minipage}[t]{.5\textwidth}
  \section{Sprachen}
  \cvline{Deutsch}{Muttersprache}
  \cvline{Englisch}{fließend}
\end{minipage}

\section{Universitäre Ausbildung}
\cventry{2010--2015}
        {Promotionsstudium in Informatik}{Westfälische Wilhelms-Universität}{Münster}{}
        {Doktorvater: Prof. Sergei Gorlatch}
\cvline{}{Titel: \textit{Improving Programmability and Performance Portability on Many-Core Processors}}
\cvline{}{Gesamtbewertung: \textit{Summa Cum Laude}}

\cventry{2005--2010}
        {Diplomstudium in Informatik mit Anwendungsfach Mathematik}
        {Westfälische Wilhelms-Universität}{Münster}{}{}
\cvline{}{Title: \textit{SkelCL - A Portable Multi-GPU Skeleton Library}}
\cvline{}{Bewertung in Informatik: \textit{1.9}}


\section{Beruflicher Werdegang}
\cventry{seit 10/2014}
        {Research Associate}{The University of Edinburgh}{Edinburgh, UK}{}{}
\cventry{08/2014\\--\ 09/2014}
        {\vspace{0em}Forschungsaufenthalt (2 Monate)}{The University of Edinburgh}{Edinburgh, UK}{}{}
\cventry{02/2014}
        {Forschungsaufenthalt (1 Monat)}{The University of Edinburgh}{Edinburgh, UK}{}{}
\cventry{07/2013\\--\ 11/2013}
        {Forschungsaufenthalt (4 Monate)}{The University of Edinburgh}{Edinburgh, UK}{}
        {Gefördert durch das EU HiPEAC Network of Excellence}
\cventry{07/2012\\--\ 10/2012}
        {Forschungsaufenthalt (3 Monate)}{The University of Edinburgh/EPCC}{Edinburgh, UK}{}
        {Gefördert durch das HPC-Europa2 Projekt}
\cventry{2010--\ 2014}
        {Wissenschaftlicher Mitarbeiter}{Westfälische Wilhelms-Universität}{}{}{}
\cventry{2008--\ 2010}
        {Studentische Hilfskraft}{Westfälische Wilhelms-Universität}{}{}{}

\section{Community Aktivitäten}
\cvline{}{Ich bin als externer Reviewer aktiv für die folgenden Konferenzen: CGO, Euro-Par, EuroMPI, CCGrid, PARCO, und
          Parallel Computing Technologies~(PaCT).
          Für das
          Science of Computer Programming journal und das
          Journal of Supercomputing.
          Sowie für den HLPGPU2012 HiPEAC Workshop.}
\cvline{}{Ich bin zurzeit Mitglied des Programm Committees des 9th International Symposium on High-Level Parallel Programming and Applications (HLPP 2016) und des 16th IEEE International Conference on Scalable Computing and Communications (ScalCom 2016).}
\cvline{}{Ich organisiere zurzeit die 7th UK Many-Core Developer Conference in Mai 2016 in Edinburgh.}


% \section{Research Projects}
% \cventry{since 2011}
%         {dOpenCL}{An implementation of the OpenCL standard targeting distributed systems}{}{}
%         {Ongoing research. Preliminary results published in: \cite{KeSG:13,KeSG:12}.}
% \cvline{}{Together with other research groups we are leading the development of
%           OpenCL implementations targeting distributed systems, by developing
%           dOpenCL. dOpenCL allows to program all parallel processors
%           (e.g., CPUs or GPUs) of a distributed Systems using OpenCL as the
%           single programming model.
%         }
%
% \cventry{since 2010}
%         {SkelCL}{A high-level programming library for heterogeneous systems}{}{}
%         {Ongoing research. Preliminary results published in: \cite{StHBG:14,StG:14,StFAG:14,GoS:14,BrSG:14,StG:13,StG:13b,StKG:12b,StKG:12,StGBB:12,KeGEDSK:13,KeSG:13b}.}
% \cvline{}{I am the lead developer of SkelCL, a novel high-level programming
%           model and library for programming heterogeneous systems. It uniquely
%           combines algorithmic skeletons, container data types and data
%           (re)distribution mechanisms to greatly simplify the programming of
%           heterogeneous systems comprising of multiple parallel processors.
%           SkelCL is open source software and can be found online at:
%           \url{http://skelcl.uni-muenster.de}.
%         }
%
% \cventry{01/2010\\ --\ 09/2010}
%         {Diploma project}{Developing a Portable Multi-GPU Skeleton Library}{}{}
%         {Results published in \cite{StKG:11}.}
% \cvline{}{As my diploma project I developed the predecessor to SkelCL,
%           an innovative high-level programming library for simplified
%           programming of GPUs. We published the results of my diploma project in
%           \cite{StKG:11} and showed, that we can greatly simplify the
%           programming of GPU systems without scarifying performance.
%           }


\printbibheading[title={Publikationen}]

\printbibliography[prefixnumbers={T},keyword=thesis,title={Dissertation},
                   heading=subbibliography]

\printbibliography[prefixnumbers={J},keyword=journal,title={Beiträge in Zeitschriften},
                   heading=subbibliography]

\printbibliography[prefixnumbers={C},keyword=conference,
                   title={Beiträge in Konferenzen},heading=subbibliography]

\printbibliography[prefixnumbers={W},keyword=workshop,
                   title={Beiträge in Workshops},heading=subbibliography]

\printbibliography[prefixnumbers={B},keyword=book,title={Kapitel in Büchern},
                   heading=subbibliography]


\section{Vorträge und Präsentationen}
  \cvline{01/2016}
         {Eingeladener Vortrag: \emph{Generating Performance Portable Code using Rewrite Rules}\newline
         \small Imperial College London, UK.}
  \cvline{12/2015}
         {Vortrag: \emph{Functional Programming in C++}\newline
         \small Programming Language Interest Group at Edinburgh University, UK.}
  \cvline{10/2015}
         {Eingeladener Vortrag: \emph{Generating Performance Portable Code using Rewrite Rules}\newline
         \small PENCIL Developer Meeting at Imperial College London, UK.}
  \cvline{10/2015}
         {Guest Lecture:\newline \emph{DSLs and rewriting-based optimizations for performance-portable parallel programming}\newline
         \small in the Elements of Programming Languages Course at the University of Edinburgh, UK.}
  \cvline{09/2015}
         {Vortrag: \emph{Generating Performance Portable Code using Rewrite Rules:\newline From High-Level Functional Expressions to High-Performance OpenCL Code}\newline
         \small International Conference on Functional Programming (ICFP) 2015 in Vancouver, Canada.}
  \cvline{06/2015}
         {Vortrag: \emph{Generating Performance Portable Code using Rewrite Rules}\newline
         \small Scottish Programming Languages Seminar in St. Andrews, UK.}
  \cvline{05/2014}
         {Eingeladener Vortrag: \emph{SkelCL: High-Level Programming of Multi-GPU
          Systems}\newline \small Institute for Computational and Applied
          Mathematics, University of Münster, Germany.}
  \cvline{05/2014}
         {Eingeladener Vortrag: \emph{SkelCL: High-Level Programming of Multi-GPU
          Systems}\newline \small Workshop on Fast Data Processing on GPUs in
          Dresden, Germany.}
  \cvline{01/2014}
         {Vortrag: \emph{Extending the SkelCL Library for Stencil
          Computations on Multi-GPU Systems}\newline \small HiStencils 2014
          workshop in Vienna, Austria.}
  \cvline{12/2013}
         {Eingeladener Vortrag: \emph{SkelCL: High-Level Programming of Multi-GPU
          Systems}\newline \small Research group on elementary particle physics,
          University of Wuppertal, Germany.}
  \cvline{07/2013}
         {Vortrag: \emph{Introducing and Implementing the Allpairs Skeleton for GPU
          Systems}\newline \small HLPP 2013 workshop in Paris, France.}
  \cvline{06/2013}
         {Vortrag:\emph{High-Level Programming for Medical Imaging on Multi-GPU
          Systems\newline using the SkelCL Library}\newline \small ICCS 2013 conference in
          Barcelona, Spain.}
  \cvline{08/2012}
       {Vortrag: \emph{Using the SkelCL Library for High-Level GPU Programming of
        2D Applications}\newline \small ParaPhrase 2012 workshop held in
        conjunction with Euro-Par 2012 in Rhodes, Greece.}
  \cvline{06/2012}
       {Vortrag: \emph{High-Level Programming for Heterogeneous Systems with
        Accelerators}\newline \small PDESoft 2012 workshop in Münster, Germany.}
  \cvline{05/2012}
       {Vortrag:\emph{Towards High-Level Programming of Multi-GPU Systems Using
        the SkelCL Library}\newline \small AsHES 2012 workshop held in
        conjunction with IPDPS 2012 in Shanghai, China.}
  \cvline{04/2012}
       {Eingeladener Vortrag: \emph{A Skeleton Library for Heterogeneous
        Multi-/Many-Core Systems}\newline \small NAIS workshop in Edinburgh, UK.}
  \cvline{01/2012}
       {Vortrag: \emph{Towards a High-Level Approach for Programming Distributed
        Systems with GPUs}\newline \small COST Action IC0805 (``ComplexHPC'')
        meeting in Timisoara, Romania.}
  \cvline{12/2011}
       {Eingeladener Vortrag: \emph{SkelCL -- A High-Level Programming Library for GPU
        Programming}\newline \small Jülich Supercomputing Centre (JSC), Germany.}
  \cvline{05/2011}
       {Vortrag: \emph{SkelCL -- A Portable Skeleton Library for High-Level
        GPU Programming}\newline
        \small HIPS 2011 workshop held in conjunction with IPDPS 2011 in
        Anchorange, Alaska, USA.}
  \cvline{09/2008}
       {Eingeladener Vortrag: \emph{Development of an Online Game as a Student Project}\newline
        \small ITSoftTEAM workshop in Chernihiv, Ukraine.}



\section{Lehre}
\cvline{SoSe 2014}{
        Betreuung eines Studentenprojekts:
        \emph{Entwurf und Implementierung einer high-level API zur Programmierung heterogener Cluster Systeme}.
      }
\cvline{WiSe 2013/2014}{
        Betreuung eines Studentenprojekts:
        \emph{High-level Programmierung von Online Spielen in Netzwerken der nächsten Generation}.
      }
\cvline{SoSe 2013}{
        Dozent für den Kurs:
        \emph{Einführung in die Programmierung mit C und C++}.\newline
        Lehrassistent für den Kurs:
        \emph{Multi-core and GPU: Parallele Programmierung}.
      }
\cvline{WiSe 2011/2012}{
        Lehrassistent für den Kurs:
        \emph{Betriebssysteme}.
      }
\cvline{SoSe 2012}{
        Betreuung eines Studentenprojekts: 
        \emph{High-level Programmierung von heterogenen parallelen Systemen}.\newline
        Lehrassistent für den Kurs:
        \emph{Multi-core and GPU: Parallele Programmierung}.
      }
\cvline{WiSe 2011/2012}{
        Lehrassisten für das Seminar:
        \emph{Technische Aspekte des Cloud Computings}.\newline
        Lehrassistent für den Kurs:
        \emph{Betriebssysteme}.
      }
\cvline{SoSe 2011}{
        Betreuung eines Studentenprojekts:
        \emph{Internet- und GPU-basiertes Cloud Computing}.\newline
        Lehrassistent für den Kurs:
        \emph{Multi-core and GPU: Parallele Programmierung}.
      }
\cvline{WiSe 2010/2011}{
        Betreuung eines Studentenprojekts:
        \emph{High-level GPU Programmierung}.
      }

\section{Betreute Studenten}
\cvline{}{\footnotesize Die folgenden Studenten werden alle zusammen mit Prof. Christophe Dubach betreut.}
\cvline{seit 09/2015}{MSc and PhD studies of Daniel Hillerström}
\cvline{seit 10/2014}{MSc and PhD studies of Adam Harries}
\cvline{seit 10/2014}{PhD studies of Toomas Remmelg}
\cvline{seit 10/2014}{PhD studies of Juan Jos{\'{e}} Fumero}
\cvline{}{\footnotesize Die folgenden Studenten wurden alle zusammen mit Prof. Sergei Gorlatch betreut.}
\cvline{07/2014}{Bachelorarbeit von André Lüers:
                 \emph{Evaluation der Skelettbibliothek FastFlow}}
\cvline{07/2014}{Bachelorarbeit von Lars Klein:
                 \emph{Eine parallele Implementierung der T-CUP Software mithilfe der SkelCL-Bibliothek}}
\cvline{01/2014}{Masterarbeit von Michael Olejnik:
                       \emph{Ein GPU-basiertes Klassifikations-Framework zur HIV Resistenzvorhersage}
                       \footnotesize (Betreut zusammen mit Dr. habil. Dominik Heider)}
\cvline{01/2014}{Masterarbeit von Stefan Breuer:
                       \emph{Erweiterung der SkelCL-Bibliothek mit einem Skelett für Stencil Berechnungen}}
\cvline{11/2013}{Diplomarbeit von Wadim Hamm:
                       \emph{Entwicklung eines Divide \& Conquer Skelettes für SkelCL}}
\cvline{07/2013}{Bachelorarbeit von Matthias Droste:
                       \emph{Evaluation der Skelettbibliothek SkePU}}
\cvline{06/2013}{Bachelorarbeit von Kai Kientopf:
                       \emph{Implementierung des Needleman-Wunsch Algorithmus und der Breitensuche mit der SkelCL-Bibliothek}}
\cvline{06/2013}{Masterarbeit von Florian Quinkert:
                       \emph{Entwicklung eines Modells zur Vorhersage von Arbeitsverteilung in heterogenen Systemen und seine Implementierung in der SkelCL-Bibliothek}}
\cvline{03/2013}{Masterarbeit von Malte Friese:
                 \emph{Ergänzung der Skelett-Bibliothek SkelCL um ein Skelett für All-Pairs-Berechnungen}}
\cvline{03/2013}{Bachelorarbeit von Sebastian Mißbach:
                 \emph{Implementierung der LR-Zerlegung und des Mersenne-Twister mit der SkelCL-Bibliothek}}
\cvline{03/2013}{Bachelorarbeit von Patrick Schiffler:
                 \emph{Performanceanalyse von SkelCL mittels B+-Baum Traversierung und 3D Jacobi Stencil}}
\cvline{01/2013}{Diplomarbeit von Markus Blank-Burian:
                 \emph{Simulation und Analyse zweidimensionaler Turbulenz auf parallelen Rechnerarchitekturen}
                 \footnotesize (Betreut zusammen mit Prof. Gernot Münster)}
\cvline{06/2012}{Diplomarbeit von Matthias Buß:
                 \emph{Erweiterung der SkelCL-Bibliothek um mehrdimensionale Datentypen}}
\cvline{09/2011}{Bachelorarbeit von Michael Olejnik:
                 \emph{Untersuchung des Einsatzes von Grafikkarten für den Radixsort}}
\cvline{09/2011}{Bachelorarbeit von Jan Gerd Tenberge:
                 \emph{Eine Erweiterung der SkelCL-Bibliothek für GPUs mit Iteratoren}}
\cvline{08/2011}{Bachelorarbeit von Stefan Breuer:
                 \emph{Verbesserung des MapOverlap Skelettes in SkelCL}}
\cvline{08/2011}{Bachelorarbeit von Tobias Günnewig:
                 \emph{Entwicklung einer Bibliothek zur Manipulation und Interpretation von Quellcode der Sprachen der C-Familie}
                 \footnotesize (Betreut zusammen mit Dr. Philipp Kegel)}

\end{document}


%% end of file `template_en.tex'.
