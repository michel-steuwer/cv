\documentclass[11pt,a4paper]{moderncv}

\renewcommand*{\familydefault}{\sfdefault}


\urlstyle{same}

\newboolean{eng}
\setboolean{eng}{true}
\newboolean{ger}
\setboolean{ger}{false}

\newcommand{\lang}[2]
{
  \ifthenelse{\boolean{eng}}{#1}{}%
  \ifthenelse{\boolean{ger}}{#2}{}
}

\usepackage{xstring}
\def\FormatName#1{%
  \IfSubStr{#1}{Steuwer}{\textbf{#1}}{#1}%
}

\moderncvtheme[blue]{classic}
\usepackage[utf8]{inputenc}
% adjust the page margins
\usepackage[scale=0.85]{geometry}
\AtBeginDocument{\recomputelengths}

% personal data
\firstname{Michel}
\familyname{Steuwer}
\lang{\address{Coesfeldweg 79a}{48161 Münster, Germany}}
     {\address{Coesfeldweg 79a}{48161 Münster, Deutschland}}
\phone{+49 (0)251 83-32744}
\fax{+49 (0)251 83-32742}
\email{michel.steuwer@uni-muenster.de}

% divide publications 
\usepackage[style=numeric-comp,
            sorting=none, % keep order as in the bib file ...
            defernumbers,maxbibnames=50]{biblatex}

\defbibenvironment{bibliography}
  {\list
     {\printfield{year}\hspace{1em}\printtext[labelnumberwidth]{\printfield{prefixnumber}\printfield{labelnumber}}}
     {\setlength{\topsep}{0pt}% layout parameters from moderncvstyleclassic.sty
      \setlength{\labelwidth}{\hintscolumnwidth}%
      \setlength{\labelsep}{\separatorcolumnwidth}%
      \leftmargin\labelwidth%
      \advance\leftmargin\labelsep}%
      \sloppy\clubpenalty4000\widowpenalty4000}
  {\endlist}
  {\item}

% Highlight my Name in bold
\DeclareNameFormat{author}{%
\ifthenelse{\equal{#1}{Steuwer}}%
    {{\bfseries\ifblank{#4}{}{#4\space}#1}}%
    {\ifblank{#4}{}{#4\space}#1}%
\ifthenelse{\value{listcount}<\value{liststop}}%
    {\addcomma\space}
    {}}

% Only print a year once
\newcounter{currentYear}
\DeclareFieldFormat{year}{%
\ifthenelse{\equal{#1}{\arabic{currentYear}}}%
    {}
    {\setcounter{currentYear}{#1}{\bfseries #1}}}

\bibliography{all}

%----------------------------------------------------------------------------------
%            content
%----------------------------------------------------------------------------------
\begin{document}
\nocite{*} % cite everything ...
\maketitle
\vspace{-2em}

\section{Personal Details}
\cvline{Birthday}{21 May 1985}
\cvline{Birthplace}{Duisburg, Germany}
\cvline{Nationality}{German}

\section{University Education}
\cventry{2010--2014\\(expected)}
        {Ph.\,D. studies}{University of Münster}{Münster, Germany}{}
        {Supervisor: Prof. Sergei Gorlatch}
\cvline{}{Main research interests:
          High-level abstractions for parallel programming using patterns,
          Exploiting modern parallel processors, including multi-core CPUs and GPUs.}

\cventry{2005--2010}
        {Diploma degree in computer science with a minor in mathematics}
        {(equivalent to a M.\,Sc. degree) University of Münster}{Münster, Germany}{}
        {Final grade in computer science: very good (85 \%)}


\section{Research Projects}
\cventry{since 2010}
        {SkelCL}{A high-level programming library for heterogeneous systems}{}{}
        {Ongoing research. Preliminary results published in: \cite{StG:14,StFAG:13,GoS:14,BrSG:14,StG:13,StG:13b,StKG:12b,StKG:12,StGBB:12}.}
\cvline{}{I am the lead developer of SkelCL, a novel high-level C++ library for programming
          heterogeneous systems. It uniquely combines parallel patterns, container data
          types and data (re)distribution mechanisms to greatly simplify the programming
          of heterogeneous systems comprising of multiple parallel processors.
          SkelCL is open source software and can be found online at:\newline
          \url{http://skelcl.uni-muenster.de}.
        }
        
\cventry{since 2011}
        {dOpenCL}{An implementation of the OpenCL standard targeting distributed systems}{}{}
        {Ongoing research. Preliminary results published in: \cite{KeSG:13,KeSG:12}.}
\cvline{}{The dOpenCL library is an OpenCL implementation targeting distributed
          systems. It allows to program all parallel processors
          (e.g., CPUs or GPUs) of a distributed Systems using OpenCL as the
          single programming model.
          dOpenCL is open source software and can be found online at:
          \url{http://dopencl.uni-muenster.de}.
        }

\cventry{01/2010\\ --\ 09/2010}
        {Diploma project}{Developing a Portable Multi-GPU Skeleton Library}{}{}
        {Results published in \cite{StKG:11}.}
\cvline{}{As my diploma project I developed the predecessor to SkelCL,
          an innovative high-level programming library for simplified
          programming of GPUs. We published the results of my diploma project in
          \cite{StKG:11} and showed, that we can greatly simplify the
          programming of GPU systems without scarifying performance.
          }

\section{Research Visits}
\cventry{02/2014}
         {Visiting researcher (1 Month)}{Edinburgh University}{Edinburgh, UK}{}
         {}
\cventry{07/2013\\--\ 11/2013}
        {Visiting researcher (4 Month)}{Edinburgh University}{Edinburgh, UK}{}
        {Funded by the HiPEAC Network of Excellence}
\cventry{07/2012\\--\ 10/2012}
        {Visiting researcher (3 Month)}{EPCC (Edinburgh Parallel Computing Centre)}{Edinburgh, UK}{}
        {Funded by the HPC-Europa2 project}

\printbibheading[title={Publications}]

\printbibliography[prefixnumbers={J},keyword=journal,title={Journal Articles},
                   heading=subbibliography]

\printbibliography[prefixnumbers={C},keyword=proceedings,
                   title={Conference Proceedings},heading=subbibliography]

%\printbibliography[prefixnumbers={B},keyword=book,title={Book Chapter},
%                   heading=subbibliography]

\end{document}

