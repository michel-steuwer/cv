%%%%%%%%%%%%%%%%%%%%%%%%%%%%%%%%%%%%%%%%%
% Medium Length Professional CV
% LaTeX Template
% Version 2.0 (8/5/13)
%
% This template has been downloaded from:
% http://www.LaTeXTemplates.com
%
% Original author:
% Trey Hunner (http://www.treyhunner.com/)
%
% Important note:
% This template requires the resume.cls file to be in the same directory as the
% .tex file. The resume.cls file provides the resume style used for structuring the
% document.
%
%%%%%%%%%%%%%%%%%%%%%%%%%%%%%%%%%%%%%%%%%

%----------------------------------------------------------------------------------------
%	PACKAGES AND OTHER DOCUMENT CONFIGURATIONS
%----------------------------------------------------------------------------------------

\documentclass[letterpaper]{resume} % Use the custom resume.cls style

\usepackage[nochapters]{classicthesis} % Use the classicthesis style for the style of the document

\usepackage[left=0.75in,top=0.6in,right=0.75in,bottom=0.6in]{geometry} % Document margins

\usepackage{multicol}

\name{Michel Steuwer} % Your name
\address{\small Coesfeldweg 79a, 48161 M\"unster, Germany \hfill +49 251 8332744 \hspace{2em} michel.steuwer@uni-muenster.de} % Your address

\begin{document}

\begin{rSection}{Education}

\begin{rSubsection}{University of Edinburgh\normalfont{}, UK}{07/2013 -- 11/2013}%
                    {Visiting Researcher funded by the HiPEAC Network of Excellence}{}
  \item \vspace{-1.5em}
\end{rSubsection}

\begin{rSubsection}{University of Edinburgh\normalfont{}, UK}{07/2012 -- 10/2012}%
                   {Visiting Researcher funded by the HPC-Europa2 project}{}
\item \vspace{-1.5em}
\end{rSubsection}

\begin{rSubsection}{University of M\"unster\normalfont{}, Germany}{2010 -- present}%
                   {PhD in Computer Science}{}
  \item \emph{Research interests:} GPU Programming, Algorithmic Skeletons,
      Code Generation, Performance Portability
    \begin{itemize}
      \item[] I'm interested in researching technologies and techniques that help
        application developers to write programs at a high-level of abstraction
        while achieving high-performance at the same time.
        
        My work focuses on recurring patterns of parallel programming,
        a.k.a. algorithmic skeletons,
        and how these can be efficiently be implemented on modern micro
        architectures like GPUs.
        
        %More complex algorithms can then be composed
        %from the existing skeletons.
        %Manually providing high-performance implementations for different
        %architectures is elaborate and tedious, therefore, I'm currently
        %researching a technique to automatically generate efficient
        %skeleton implementations on different architectures.
    \end{itemize}
\end{rSubsection}

\begin{rSubsection}{University of M\"unster\normalfont{}, Germany}{2005 -- 2010}%
                   {Diploma in Computer Science (equivalent to a MSc. degree)}{}
  \item \emph{Final Grade in Computer Science:} very good (best possible grade)\\[-1em]
  \item \emph{Diploma Thesis:} ``Developing a Portable Multi-GPU Skeleton Library''
    \vspace{-.25em}
    \begin{itemize}
      \item[] As my diploma project I developed an innovative high-level
        programming library for simplified programming of GPUs, which was later
        extended into the SkelCL research project.


        The results of my diploma project showed,
        that we can greatly simplify the programming of GPU systems without
        sacrificing performance.
    \end{itemize}
\end{rSubsection}

\end{rSection}

\begin{rSection}{Research Projects}

\begin{rSubsection}{Skeleton Building Blocks}
                   {\hspace{-5em}\emph{Automatic GPU-Code Generation from Skeleton Building Blocks}\hfill}{}{}
  \item In this joined work with Christophe Dubach and Christian Fensch we
    build a unique system to automatically generate efficient OpenCL
    implementations from high-level expressions.
    
    By uniquely bridging algorithmic patterns with hardware paradigms using
    a rule rewriting system, we can automatically explore different
    implementations of the same algorithm.

    Results show, that we can match the performance of hand written OpenCL
    as well as highly tuned BLAS code on different hardware architectures.

    This work is currently under review for publication at the PLDI 2014
    conference.
\end{rSubsection}

\begin{rSubsection}{SkelCL}%
                   {\hspace{-10em}\emph{A High-Level Programming Model for Single- and Multi-GPU Systems}\hfill}%
                   {}{}
  \item I'm the lead designer and developer of SkelCL, a novel high-level
    programming model and library for programming heterogeneous systems.


    It combines algorithmic skeletons, container data types and data
    (re)distribution mechanisms to greatly simplify the programming of
    heterogeneous systems comprising of multiple parallel processors, \emph{e.\,g.}, multiple GPUs.


    SkelCL is open source software and available at: \url{http://github.com/skelcl/skelcl}.
\end{rSubsection}

\begin{rSubsection}{dOpenCL}
                   {\hspace{-10em}\emph{An Implementation of the OpenCL Standard for Distributed Systems}\hfill}{}{}
  \item Together with Philipp Kegel, I developed dOpenCL, an OpenCL
    implementation targeting distributed systems. dOpenCL allows to program all
    parallel processors (e.g., CPUs or GPUs) of a distributed System using OpenCL
    as the single programming model.
\end{rSubsection}

\end{rSection}

\begin{rSection}{Awards}

\begin{rSubsection}{HiPEAC Collaboration Grant}{2013}
                   {Awarded by the HiPEAC Network of Excellence}{}
  \item Funding for a four month research collaboration with Christophe Dubach at Edinburgh University.
    Prestigious award received by only 20 students per year in entire Europe.
\end{rSubsection}

\begin{rSubsection}{HPC-Europa 2 Transnational Access Programme}{2012}
                   {Awarded by the HPC-Europa 2 Project}{}
  \item Funding for a three month research collaboration with Thibaut Lutz and Murray Cole at Edinburgh University.
\end{rSubsection}
\vspace{3em}

\end{rSection}

\begin{rSection}{Publications %{\footnotesize\normalfont (out of 10 in total)}
  }

\begin{itemize}
%   \item Michel Steuwer, Christophe Dubach\\
%     \emph{Pattern Composition and Rewrite Rules for High-Performance Code Generation on Heterogeneous Systems},\\
%     submitted to the conference on Programming Language Design and Implementation (PLDI) 2014.
%     \begin{itemize}
%       \item By uniquely bridging algorithmic patterns and hardware paradigms using a rule rewriting system, 
%         we designed a system capable of automatically exploring different implementations of the same algorithm.
%         We demonstrate, that this approach is capable of achieving the same, or even better, performance
%         than hand-written implementations and highly tuned BLAS code, using reduction as our application study.
%     \end{itemize}

  \item Michel Steuwer, Malte Friese, Sebastian Albers, and Sergei Gorlatch\\
    \emph{Introducing and Implementing the Allpairs Skeleton for GPU Systems},\\
    in International Journal of Parallel Programming (in press, available
    online).\\
    Presented at the int'l. Symposium on High-level Parallel Programming and Applications, July 2013.

  \item Michel Steuwer and Sergei Gorlatch\\
    \emph{High-Level Programming for Medical Imaging on Multi-GPU Systems using
      the SkelCL Library},\\
    presented at the International Conference on Computational Science (ICCS),
    June 2013.

  \item Philipp Kegel, Michel Steuwer, and Sergei Gorlatch\\
    \emph{dOpenCL: Towards uniform programming of distributed heterogeneous
      multi-/many-core systems},\\
    in Journal of Parallel and Distributed Computing, 73(12), 2013.

  %\item Michel Steuwer and Sergei Gorlatch\\
  %  \emph{SkelCL: Enhancing OpenCL for High-Level Programming of Multi-GPU
  %    Systems},\\
  %  paper at the conference on Parallel Computing Technologies (PaCT),
  %  September 2013.

  \item Michel Steuwer, Philipp Kegel, and Sergei Gorlatch\\
    \emph{Towards High-Level Programming of Multi-GPU Systems using the
      SkelCL Library},\\
    presented at the workshop on Acccelerators and Hybrid Exascale Systems as
    part of IPDPS, May 2012.

  %\item Michel Steuwer, Philipp Kegel, and Sergei Gorlatch\\
  %  \emph{A High-Level Programming Approach for Distributed Systems with
  %    Accelerators},\\
  %  paper at the conference on Intelligent Software Methodologies, Tools and
  %  Techniques (SoMeT), September 2012.

  %\item Michel Steuwer, Sergei Gorlatch, Matthias Bu{\ss}, and Stefan Breuer\\
  %  \emph{Using the SkelCL Library for High-Level GPU Programming of 2D
  %    Applications},\\
  %  presented at the ParaPhrase workshop as part of the Euro-Par conference,
  %  August 2012.

  %\item Philipp Kegel, Michel Steuwer, and Sergei Gorlatch\\
  %  \emph{dOpenCL: Towards a Uniform Programming Approach for Distributed
  %    Heterogeneous Multi-/Many-Core Systems},\\
  %  paper at the Heterogeneity in Computing Workshop as part of the 26th
  %  International Parallel and Distributed Processing Symposium (IPDPS), May
  %  2012.

  \item Michel Steuwer, Philipp Kegel, and Sergei Gorlatch\\
    \emph{SkelCL -- A Portable Skeleton Library for High-Level GPU
      Programming},\\
    presented at the workshop on High-Level Parallel Programming Models and
    Supportive Environments as part of IPDPS, May 2011.
\end{itemize}

\end{rSection}
\vspace{3em}
\begin{rSection}{References}

\begin{center}
  \begin{tabular}{ll}
    \begin{minipage}[t]{.45\linewidth}
      {\bf Prof. Sergei Gorlatch}\\
      University of M\"unster\\[.5em]
      PhD Supervisor\\[1em]
      gorlatch@uni-muenster.de\\
      +49 251 83 32740\\[1em]
      Einsteinstra{\ss}e 62\\
      484149 M\"unster\\
      Germany
    \end{minipage} &
    \begin{minipage}[t]{.45\linewidth}
      {\bf Assistant Prof. Christophe Dubach}\\
      University of Edinburgh\\[2.5em]
      christophe.dubach@ed.ac.uk\\
      +44 131 650 3092\\[1em]
      Informatics Forum\\
      10 Chrichton Street\\
      EH8 9AB Edinburgh\\
      United Kingdom
    \end{minipage}
  \end{tabular}
\end{center}

\end{rSection}

\end{document}
