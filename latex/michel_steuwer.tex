\documentclass[11pt,a4paper]{moderncv}
% \usepackage{mathpazo}
\usepackage[T1]{fontenc}

\usepackage[sfdefault,scaled=.95]{FiraSans}
\usepackage[T1]{fontenc}
\renewcommand*\oldstylenums[1]{{\firaoldstyle{} #1}}

\urlstyle{same}

\usepackage{xstring}
\def\FormatName#1{%
  \IfSubStr{#1}{Steuwer}{\textbf{#1}}{#1}%
}

\moderncvtheme[blue]{classic}
\usepackage[utf8]{inputenc}
% adjust the page margins
\usepackage[scale=0.8]{geometry}
\AtBeginDocument{\recomputelengths}

% personal data
\firstname{Michel}
\familyname{Steuwer}
\address{%
  Informatics Forum --- 1.02\\
  10 Crichton Street\\
  Edinburgh EH8 9AB\\}{%
  United Kingdom}
\email{michel.steuwer@ed.ac.uk}

% divide publications
\usepackage[style=numeric-comp,
            sorting=none, % keep order as in the bib file ...
            defernumbers,maxbibnames=50]{biblatex}

\defbibenvironment{bibliography}
  {\list%
     {\printfield{year}\hspace{1em}\printtext[labelnumberwidth]{\printfield{prefixnumber}\printfield{labelnumber}}}
     {\setlength{\topsep}{0pt}% layout parameters from moderncvstyleclassic.sty
      \setlength{\labelwidth}{\hintscolumnwidth}%
      \setlength{\labelsep}{\separatorcolumnwidth}%
      \leftmargin\labelwidth%
      \advance\leftmargin\labelsep%
      }%
      \sloppy\clubpenalty4000\widowpenalty4000}
  {\endlist}
  {\item}

% Highlight my Name in bold
\DeclareNameFormat{author}{%
\ifthenelse{\equal{#1}{Steuwer}}%
    {{\bfseries\ifblank{#4}{}{#4\space}#1}}%
    {\ifblank{#4}{}{#4\space}#1}%
\ifthenelse{\value{listcount}<\value{liststop}}%
    {\addcomma\space}
    {}}

% Only print a year once
\newcounter{currentYear}
\DeclareFieldFormat{year}{%
\ifthenelse{\equal{#1}{\arabic{currentYear}}}%
    {}
    {\setcounter{currentYear}{#1}{\bfseries #1}}}

\bibliography{michel_steuwer}

\usepackage{xspace}

\newcommand{\since}{{\small since}\xspace}

%----------------------------------------------------------------------------------
%            content
%----------------------------------------------------------------------------------
\begin{document}
\nocite{*} % cite everything ...

\makecvtitle%
\vspace{-2em}
%\begin{minipage}[t]{0.5\textwidth}
%  \section{Personal Details}
%  %\cvline{Birthday}{21th of May 1985}
%  \cvline{Birthplace}{Duisburg, Germany}
%  \cvline{Nationality}{German}
%\end{minipage}
%\begin{minipage}[t]{.5\textwidth}
%  \section{Languages}
%  \cvline{German}{Native}
%  \cvline{English}{Fluent}
%\end{minipage}

\section{University Education}
\cventry{2010--2015}
        {PhD degree in computer science}{University of Münster}{Germany}{}
        {Supervisor: Prof.\ Sergei Gorlatch}
\cvline{}{Thesis: \textit{Improving Programmability and Performance Portability on Many-Core Processors}}
\cvline{}{Awarded with the highest possible grade: \textbf{Summa Cum Laude} ({\itshape\small with highest honor\/})}
\cvline{}{Nominated as one of 34 candidates from all German, Austrian, and Swiss Universities for the prize for best dissertation awarded by the German Informatics Society.}

\cventry{2005--2010}
        {Diploma degree in computer science with a minor in mathematics}
        {\newline(equivalent to a combined MSc and UG degree) University of Münster}{Germany}{}{}
\cvline{}{Thesis: \textit{SkelCL --- A Portable Multi-GPU Skeleton Library}}
% \cvline{}{Overall grade in computer science: \textit{very good (85 \%)}}


%\vspace{1em}
\section{Professional Experience}
\cventry{{\footnotesize since Oct.} 2014}
        {Postdoctoral Research Associate}{The University of Edinburgh}{UK}{}{}
\cventry{2010--2014}
        {Research Associate}{University of Münster}{Germany}{}{}

%\vspace{1em}
\section{Awarded Honours} % Auszeichnungen
\cvline{}{
  \begin{itemize}
    \item PhD thesis honoured with the highest possible grade \emph{Summa cum laude}
    \item Nominated as one of 34 candidates for the prize for best dissertation completed in 2015 in Informatics at a German, Austrian, or Swiss University.
          This highly prestigious prize is awarded annually by the German Informatics Society (GI).
    %\item HiPEAC collaboration grants (2016 and 2013) and HPC-Europa2 visitor grant (2012)
  \end{itemize}
}
\vspace{-2em}

\section{Awarded Grants}
\cvline{}{
  \begin{itemize}
    \item HiPEAC collaboration grants (2016 and 2013) and HPC-Europa2 visitor grant (2012) in total of approx. €15.000.
    \item Nvidia GPU Grant Program (2011 and 2016) in total of approx. €12.500.
    \item Intel Hardware Accelerator Research Program (2016) for privileged access to Intel's upcoming CPU+FPGA hardware.
  \end{itemize}
}
\vspace{-2em}


\section{Research Visits and Collaborations}
\cventry{2016}{Research Collaboration (3 Month)}{dividiti Ltd.}{UK}{}
        {Funded by the HiPEAC Network of Excellence}
\cventry{2016}{Hosting of a visiting researcher (2 Month)}{From the University of Münster}{Germany}{}
        {Funded by the EuroLab-4-HPC project}
\cventry{2014}{Visiting researcher (3 Month)}{The University of Edinburgh}{UK}{}{}
\cventry{2013}
        {Visiting researcher (4 Month)}{The University of Edinburgh}{UK}{}
        {Funded by the EU HiPEAC Network of Excellence}
\cventry{2012}
        {Visiting researcher (3 Month)}{The University of Edinburgh/EPCC}{UK}{}
        {Funded by the EU HPC-Europa2 project}

\section{Research Community Activities}
\subsection{Memberships and Participation in Research Networks}
\cvline{}{
  \begin{itemize}
    \item Member of ACM and the German Informatics Society ({\footnotesize GI\@: Gesellschaft f{\"u}r Informatik})
    \item Active participating member of the European Network on High Performance and Embedded Architecture and Compilation (HiPEAC)
    \item I represent the University of Edinburgh in the recent EU \emph{EuroLab-4-HPC\@: Open source in high performance computing} initiative
  \end{itemize}
}
\subsection{Conference Organisation}
\cvline{}{
  \begin{itemize}
    \item I was the main organiser of the 7{\small th} UK Many-Core Developer Conference on May 10{\small th} in Edinburgh with over 50 participants, a keynote and 10 talks spanning topics from the landscape of accelerated, heterogeneous and many-core computing.
  \end{itemize}
}
\subsection{Organisation of Informal Groups}
\cvline{}{
  \begin{itemize}
    \item I co-organise the \emph{Programming Languages Interest Group} at the School of Informatics in Edinburgh together with James Cheney, a group discussing a broad range of topics related to programming languages.
    \item I organise the \emph{Humble C++ Programmer Group}, a group discussing practical programming topics in C++ targeted at PhD students to improve their coding skills.
  \end{itemize}
}
\subsection{Program Committees}
\cvline{}{
  \begin{itemize}
    \item {\small 10{\footnotesize th} Int. Symposium on High-Level Parallel Programming and Applications (HLPP\,2017)}
    \item {\small 9{\footnotesize th} Int. Symposium on High-Level Parallel Programming and Applications (HLPP\,2016)}
    \item {\small 16{\footnotesize th} IEEE Int. Conference on Scalable Computing and Communications (ScalCom\,2016)}
    % \item {\small International Workshop on Applications of Language Engineering Techniques in Many-Core Compilers (ALLIANCE 2016)}
\end{itemize}
}
\subsection{Artifact Evaluation Committees}
\cvline{}{
  \begin{itemize}
    \item {The \small 25{\footnotesize th} Int. Conference on Parallel Architectures and Compilation Techniques (PACT\,2016)}
    \item {The \small 15{\footnotesize th} Int. Symposium on Code Generation and Optimization (CGO\,2017)}
  \end{itemize}
}
\subsection{Reviewing}
\cvline{}{External reviewer for conferences:
  \begin{itemize}
    \item International Symposium on Code Generation and Optimization (CGO)
    \item International Conference on Parallel and Distributed Computing (Euro-Par)
    \item European MPI Users Group conference (EuroMPI)
    \item International Symposium on Cluster, Cloud and Grid Computing (CCGrid)
    \item International Parallel Computing Conference (ParCo)
    \item Parallel Computing Technologies (PaCT)
  \end{itemize}
  \smallskip
  Reviewer for journals:
  \begin{itemize}
    \item ACM Computing Surveys (ACM)
    \item Science of Computer Programming Journal (Elsevier)
    \item The Journal of Supercomputing (Springer)
    \item Software: Practice and Experience (Wiley)
  \end{itemize}
}
          %the Science of Computer Programming journal, the Software:
          %Practice and Experience journal and the
          %Journal of Supercomputing;
          %as well as the HLPGPU HiPEAC Workshop.}


% TODO: check names
\section{Research Collaborations}
\cvline{}{
  \begin{itemize}
    \item Sam Lindley, LFCS, University of Edinburgh
    \item Alan Gray, EPCC, University of Edinburgh
    \item Robert Atkey, University of Strathclyde
    \item Ryan Newton, University of Indiana Bloomington
    \item Sergei Gorlatch, University of M{\"u}nster
    \item Alastair Murray, Codeplay
    \item Grigori Fursin and Anton Lokhmotov, dividiti Ltd.
    \item Mario Wolczko and Tim Harris, Oracle Labs
    \item Robert Hundt, Google
  \end{itemize}
}

% \section{Research Projects}
% \cventry{since 2011}
%         {dOpenCL}{An implementation of the OpenCL standard targeting distributed systems}{}{}
%         {Ongoing research. Preliminary results published in: \cite{KeSG:13,KeSG:12}.}
% \cvline{}{Together with other research groups we are leading the development of
%           OpenCL implementations targeting distributed systems, by developing
%           dOpenCL. dOpenCL allows to program all parallel processors
%           (e.g., CPUs or GPUs) of a distributed Systems using OpenCL as the
%           single programming model.
%         }
%
% \cventry{since 2010}
%         {SkelCL}{A high-level programming library for heterogeneous systems}{}{}
%         {Ongoing research. Preliminary results published in: \cite{StHBG:14,StG:14,StFAG:14,GoS:14,BrSG:14,StG:13,StG:13b,StKG:12b,StKG:12,StGBB:12,KeGEDSK:13,KeSG:13b}.}
% \cvline{}{I am the lead developer of SkelCL, a novel high-level programming
%           model and library for programming heterogeneous systems. It uniquely
%           combines algorithmic skeletons, container data types and data
%           (re)distribution mechanisms to greatly simplify the programming of
%           heterogeneous systems comprising of multiple parallel processors.
%           SkelCL is open source software and can be found online at:
%           \url{http://skelcl.uni-muenster.de}.
%         }
%
% \cventry{01/2010\\ --\ 09/2010}
%         {Diploma project}{Developing a Portable Multi-GPU Skeleton Library}{}{}
%         {Results published in \cite{StKG:11}.}
% \cvline{}{As my diploma project I developed the predecessor to SkelCL,
%           an innovative high-level programming library for simplified
%           programming of GPUs. We published the results of my diploma project in
%           \cite{StKG:11} and showed, that we can greatly simplify the
%           programming of GPU systems without scarifying performance.
%           }

\section{Talks and Presentations}
  \cvline{12/2016}
         {Invited Talk:\newline \emph{The Lift Project: Performance Portable GPU Code Generation via Rewrite Rules}\newline
         Computer Laboratory Systems Research Group Seminar, University of Cambridge, UK.
         }
  \cvline{08/2016}
         {Invited Talk:\newline \emph{Structured Parallel Programming --- From High-Level Functional Expressions to High-Performance OpenCL Code}\newline
         Center for Advanced Electornics Dresden, Dresden University of Technology, Germany.
         }
  \cvline{05/2016}
         {Invited Talk:\newline \emph{Improving Programmability and Performance Portability on Many-Core Processors}\newline
         \small Colloquium of candidates nominated for the \emph{prize for best dissertation} awarded by the German Informatics Society, Schloss Dagstuhl, Germany.}
  \cvline{04/2016}
         {Invited Talk: \emph{The lift Project: Performance Portability via Rewrite Rules}\newline
          Saarland University, Germany.}
  \cvline{01/2016}
         {Invited Talk: \emph{Performance Portable GPU Code Generation}\newline
         \small Imperial College London, UK.}
  \cvline{12/2015}
         {Talk: \emph{Functional Programming in C++}\newline
         \small Programming Language Interest Group at Edinburgh University, UK.}
  \cvline{10/2015}
         {Invited Talk: \emph{Generating Performance Portable Code using Rewrite Rules}\newline
         \small PENCIL Developer Meeting at Imperial College London, UK.}
  \cvline{09/2015}
         {Talk: \emph{Generating Performance Portable Code using Rewrite Rules:\newline From High-Level Functional Expressions to High-Performance OpenCL Code}\newline
         \small International Conference on Functional Programming (ICFP) 2015 in Vancouver, Canada.}
  \cvline{06/2015}
         {Talk: \emph{Generating Performance Portable Code using Rewrite Rules}\newline
         \small Scottish Programming Languages Seminar in St.\ Andrews, UK.}
  \cvline{05/2014}
         {Invited Talk: \emph{SkelCL\@: High-Level Programming of Multi-GPU
          Systems}\newline \small Institute for Computational and Applied
          Mathematics, University of Münster, Germany.}
  \cvline{05/2014}
         {Invited Talk: \emph{SkelCL\@: High-Level Programming of Multi-GPU
          Systems}\newline \small Workshop on Fast Data Processing on GPUs in
          Dresden, Germany.}
  \cvline{01/2014}
         {Talk: \emph{Extending the SkelCL Library for Stencil
          Computations on Multi-GPU Systems}\newline \small HiStencils 2014
          workshop in Vienna, Austria.}
  \cvline{12/2013}
         {Invited Talk: \emph{SkelCL\@: High-Level Programming of Multi-GPU
          Systems}\newline \small Research group on elementary particle physics,
          University of Wuppertal, Germany.}
  \cvline{07/2013}
         {Talk: \emph{Introducing and Implementing the Allpairs Skeleton for GPU
          Systems}\newline \small HLPP 2013 workshop in Paris, France.}
  \cvline{06/2013}
         {Talk:\emph{High-Level Programming for Medical Imaging on Multi-GPU
          Systems\newline using the SkelCL Library}\newline \small ICCS 2013 conference in
          Barcelona, Spain.}
  \cvline{08/2012}
       {Talk: \emph{Using the SkelCL Library for High-Level GPU Programming of
        2D Applications}\newline \small ParaPhrase 2012 workshop held in
        conjunction with Euro-Par 2012 in Rhodes, Greece.}
  \cvline{06/2012}
       {Talk: \emph{High-Level Programming for Heterogeneous Systems with
        Accelerators}\newline \small PDESoft 2012 workshop in Münster, Germany.}
  \cvline{05/2012}
       {Talk:\emph{Towards High-Level Programming of Multi-GPU Systems Using
        the SkelCL Library}\newline \small AsHES 2012 workshop held in
        conjunction with IPDPS 2012 in Shanghai, China.}
  \cvline{04/2012}
       {Invited talk: \emph{A Skeleton Library for Heterogeneous
        Multi-/Many-Core Systems}\newline \small NAIS workshop in Edinburgh, UK.}
  \cvline{01/2012}
       {Talk: \emph{Towards a High-Level Approach for Programming Distributed
        Systems with GPUs}\newline \small COST Action IC0805 (``ComplexHPC'')
        meeting in Timisoara, Romania.}
  \cvline{12/2011}
       {Invited talk: \emph{SkelCL --- A High-Level Programming Library for GPU
        Programming}\newline \small Jülich Supercomputing Centre (JSC), Germany.}
  \cvline{05/2011}
       {Talk: \emph{SkelCL --- A Portable Skeleton Library for High-Level
        GPU Programming}\newline
        \small HIPS 2011 workshop held in conjunction with IPDPS 2011 in
        Anchorange, Alaska, USA.}
  \cvline{09/2008}
       {Invited talk: \emph{Development of an Online Game as a Student Project}\newline
        \small ITSoftTEAM workshop in Chernihiv, Ukraine.}

%\newpage

\printbibheading[title={Publications}]

\printbibliography[prefixnumbers={J},keyword=journal,title={Journal Articles},
                   heading=subbibliography]

\printbibliography[prefixnumbers={C},keyword=conference,
                   title={Conference Proceedings},heading=subbibliography]

\printbibliography[prefixnumbers={W},keyword=workshop,
                   title={Workshop Proceedings},heading=subbibliography]

\printbibliography[prefixnumbers={T},keyword=thesis,title={Thesis},
                   heading=subbibliography]

\printbibliography[prefixnumbers={B},keyword=book,title={Book Chapter},
                   heading=subbibliography]
 % \printbibliography[prefixnumbers={S},keyword=submission,title={Papers currently in submission},
%                    heading=subbibliography]




\section{Teaching Experience}
\cvline{\small Fall Term 2016}{%
  \begin{itemize}
    \item Guest Lecture on \emph{\small DSLs and rewrite-based optimizations for performance-portable parallel programming} in the \emph{Elements of Programming Languages} course held by James Cheney.
    \item Guest Lecture in the \emph{Compiling Techniques} course given by Christophe Dubach.
    \item Assistance in the tutorials of the \emph{Compiling Techniques} course held by Christophe Dubach.
  \end{itemize}
  }
\cvline{\small Fall Term 2015}{%
  \begin{itemize}
    \item Organiser and Lecturer of the C++ programming course \emph{The Humble C++ Programmer} aiming to improve PhD students coding skills.
    \item Guest Lecture on \emph{\small DSLs and rewrite-based optimizations for performance-portable parallel programming} in the \emph{Elements of Programming Languages} course held by James Cheney.
    \item Assistance in the tutorials of the \emph{Compiling Techniques} course held by Christophe Dubach.
  \end{itemize}
  }
\vspace{-1em}
\cvline{\small Fall Term 2014}{%
  \begin{itemize}
    \item Guest Lecture in the \emph{Compiling Techniques} course given by Christophe Dubach.
  \end{itemize}
}
\vspace{-1em}
\cvline{\small Summer Term 2014}{%
\begin{itemize}
  \item      Supervised MSc student project:\newline
        \emph{Design and implementation of a high-level API for programming heterogeneous clusters}.
\end{itemize}
      }
\vspace{-1em}
\cvline{\small Winter Term 2013/2014}{%
  \begin{itemize}
    \item Supervised MSc student project:\newline
        \emph{High-level programming of online games in future generation networks}.
  \end{itemize}
      }
\vspace{-1em}
\cvline{\small Summer Term 2013}{%
  \begin{itemize}
    \item  Course Design and Lecturer:
        \emph{Introduction to programming with C and C++}.
      \item    Teaching assistant:
        \emph{Multi-core and GPU\@: Parallel Programming}.
    \end{itemize}
      }
\vspace{-1em}
\cvline{\small Winter Term 2011/2012}{%
  \vspace{0em}
  \begin{itemize}
    \item Teaching assistant:
        \emph{Operating Systems}.
    \end{itemize}
      }
\vspace{-1em}
\cvline{\small Summer Term 2012}{%
  \begin{itemize}
    \item Supervised MSc student project:
        \emph{High-level programming of heterogeneous systems}.
      \item Teaching assistant:
        \emph{Multi-core and GPU\@: Parallel Programming}.
    \end{itemize}
      }
\vspace{-1em}
\cvline{\small Winter Term 2011/2012}{%
  \begin{itemize}
    \item Teaching assistant:
        \emph{Technical aspects of cloud computing}.
      \item Teaching assistant:
        \emph{Operating Systems}.
    \end{itemize}
      }
\vspace{-1em}
\cvline{\small Summer Term 2011}{%
  \begin{itemize}
    \item Supervised UG/MSc student project:
        \emph{Internet- and GPU-based Cloud Computing}.
      \item Course Design and teaching assistant:
        \emph{Multi-core and GPU\@: Parallel Programming}.
    \end{itemize}
      }
\vspace{-1em}
\cvline{\small Winter Term 2010/2011}{%
  \vspace{0em}
  \begin{itemize}
    \item Supervised UG student project:
        \emph{High-level GPU programming}.
    \end{itemize}
      }

\section{Supervised Students}
\cvline{}{\footnotesize The following students are co-supervised with Sergei Gorlatch at the University of M{\"u}nster.}
\cvline{\since{} 04/2016}{PhD studies of Bastian Hagedorn on\newline \emph{Efficient GPU Code Generation for Stencil Computations in Lift}}
\cvline{\since{} 06/2015}{PhD studies of Ari Rasch on\newline \emph{Parametric Algorithmic Skeletons}}
\cvline{\since{} 06/2015}{PhD studies of Michael Haidl on\newline \emph{{\scshape pacxx}: A GPU programming model embedded in C++}}
\cvline{}{\footnotesize The following students are co-supervised with Christophe Dubach at the University of Edinburgh.}
\cvline{\since{} 09/2015}{PhD studies of Daniel Hillerström on\newline \emph{Efficient Compilation of Handlers for Algebraic Effects}} 
\cvline{\since{} 09/2015}{PhD studies of Larisa Stoltzfus on\newline \emph{Stencil-based Acoustic Applications}}
\cvline{\since{} 09/2015}{PhD studies of Vanya Yaneva on\newline \emph{Parallel Test Exectuion on GPUs}}
\cvline{\since{} 10/2014}{PhD studies of Adam Harries on\newline \emph{Sparse and Irregular Data-Parallel Applications on GPUs}}
\cvline{\since{} 10/2014}{PhD studies of Juan Jos{\'{e}} Fumero on\newline \emph{Heterogeneous Computing in Managed Languages}}
\cvline{\since{} 10/2014}{PhD studies of Toomas Remmelg on\newline \emph{Automatic Performance Optimisations via Provably Correct Rewrite Rules}}
\cvline{}{\footnotesize The following students have been co-supervised with Sergei Gorlatch at the University of Münster.}
\cvline{09/2016}{MSc thesis of Bastian Hagedorn on\newline \emph{Efficient GPU Code Generation for Stencil Computations via Parallel Patterns}}
\cvline{07/2014}{Bachelor thesis of André Lüers on\newline
                 \emph{Evaluation of the Skeleton Library FastFlow}}
\cvline{07/2014}{Bachelor thesis of Lars Klein on\newline
                 \emph{A Parallel Implementation of the T-CUP Software using the
                       SkelCL Library}}
\cvline{01/2014}{Master thesis of Michael Olejnik on\newline
                       \emph{A GPU-based Classification Framework for
                             HIV Resistance Prediction}}
\cvline{01/2014}{Master thesis of Stefan Breuer on\newline
                       \emph{Extending the SkelCL Library for
                             Stencil Computations}}
\cvline{11/2013}{Diploma thesis of Wadim Hamm on\newline
                       \emph{Development of a Divide \& Conquer Skeleton for
                             SkelCL}}
\cvline{07/2013}{Bachelor thesis of Matthias Droste on\newline
                       \emph{Evaluation of the Skeleton Library SkePU}}
\cvline{06/2013}{Bachelor thesis of Kai Kientopf on\newline
                       \emph{\small Implementation of the Needleman-Wunsch Algorithm
                             and the Breath-First-Search with SkelCL}}
\cvline{06/2013}{Master thesis of Florian Quinkert on\newline
                       \emph{\small A Model for Predicting Work Distribution in
                             Heterogeneous Systems and its Implementation
                             in SkelCL}}
\cvline{03/2013}{Master thesis of Malte Friese on\newline
                 \emph{Extending the Skeleton Library SkelCL with a Skeleton for
                       Allpairs Computations}}
\cvline{03/2013}{Bachelor thesis of Sebastian Mißbach on\newline
                 \emph{Implementing the LU-Decomposition and the Mersenne-Twister
                       with the SkelCL Library}}
\cvline{03/2013}{Bachelor thesis of Patrick Schiffler on\newline
                 \emph{Performance Analysis of SkelCL using B+ Tree Traversal and
                       3D Jacobi Stencil Computation}}
\cvline{01/2013}{Diploma thesis of Markus Blank-Burian on\newline
                 \emph{Simulation and Analysis of Two-Dimensional Turbulences on
                       Parallel Computer Architectures}}
\cvline{06/2012}{Diploma thesis of Matthias Buß on\newline
                 \emph{Adding Multidimensional Data Types to the Multi-GPU
                       Skeleton Library SkelCL}}
\cvline{09/2011}{Bachelor thesis of Michael Olejnik on\newline
                 \emph{Investigating the Use of GPUs for Radix Sort}}
\cvline{09/2011}{Bachelor thesis of Jan Gerd Tenberge on\newline
                 \emph{Extending the SkelCL Library with Iterators}}
\cvline{08/2011}{Bachelor thesis of Stefan Breuer on\newline
                 \emph{Enhancing SkelCL's MapOverlap Skeleton}}
\cvline{08/2011}{Bachelor thesis of Tobias Günnewig on\newline
                 \emph{Developing a Library for Manipulating Source Code of
                   C-based Languages}}

\end{document}


%% end of file `template_en.tex'.
