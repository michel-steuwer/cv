\documentclass[11pt,a4paper]{moderncv}

\renewcommand*{\familydefault}{\sfdefault}


\urlstyle{same}

\newboolean{eng}
\setboolean{eng}{true}
\newboolean{ger}
\setboolean{ger}{false}

\newcommand{\lang}[2]
{
  \ifthenelse{\boolean{eng}}{#1}{}%
  \ifthenelse{\boolean{ger}}{#2}{}
}

\usepackage{xstring}
\def\FormatName#1{%
  \IfSubStr{#1}{Steuwer}{\textbf{#1}}{#1}%
}

%\moderncvtheme[blue]{casual}
\moderncvtheme[blue]{classic}
\usepackage[utf8]{inputenc}
% adjust the page margins
\usepackage[scale=0.8]{geometry}
\AtBeginDocument{\recomputelengths}

% personal data
\firstname{Michel}
\familyname{Steuwer}
\lang{\address{Coesfeldweg 79a}{48161 Münster, Germany}}
     {\address{Coesfeldweg 79a}{48161 Münster, Deutschland}}
\phone{+49 (0)251 83-32744}
\fax{+49 (0)251 83-32742}
\email{michel.steuwer@uni-muenster.de}
%\homepage{http://www.uni-muenster.de/PVS/en/mitarbeiter/steuwer.html}
%\extrainfo{\small\href{http://www.uni-muenster.de/PVS/en/mitarbeiter/steuwer.html}{http://www.uni-muenster.de/PVS/\\en/mitarbeiter/steuwer.html}}

% divide publications 
\usepackage{multibib}
\newcites{Jn,Cn,Bk}{{},{},{}}

%----------------------------------------------------------------------------------
%            content
%----------------------------------------------------------------------------------
\begin{document}
\maketitle

\lang{\section{Personal Details}}
     {\section{Personelle Angaben}}
\lang{\cvline{Birthday}
             {21 May 1985}}
     {\cvline{Geburtstag}
             {21.Mai 1985}}
\lang{\cvline{Birthplace}
             {Duisburg, Germany}}
     {\cvline{Geburtsort}
             {Duisburg, Deutschland}}
\lang{\cvline{Nationality}
             {German}}
     {\cvline{Nationalit\"at}
             {deutsch}}

% \lang{
% \section{School Education}
% }{
% \section{Schulbildung}
% }
% \lang{
% \cventry{1991--1994}
%         {Primary school}{Grundschule Theißelmannstraße}{Duisburg, Germany}{}{}
% }{
% \cventry{1991--1994}
%         {Grundschule}{Grundschule Theißelmannstraße}{Duisburg, Deutschland}{}{}
% }
% \lang{
% \cventry{1995--2004}
%         {Abitur, Secondary school}{Gesamtschule Dinslaken}{Dinslaken, Germany}{}
%         {Final grade: 1.5 (very good)}
% }{
% \cventry{1995--2004}
%         {Abitur, weiterführende Schule}{Gesamtschule Dinslaken}{Dinslaken, Deutschland}{}
%         {Abiturnote: 1.5}
% }

\lang{
\section{University Education}
}{
\section{Universitäre Bildung}
}
\lang{
\cventry{since 2010}
        {Ph.\,D. studies}{University of Münster}{Münster, Germany}{}
        {}
\cvline{}{Supervisor: Prof. Sergei Gorlatch}
\cvline{}{Main research interests:
          High-level abstractions for parallel programming, Exploiting modern
          parallel processors, including multi-core CPUs and GPUs.}
}{
\cventry{seit 2010}
        {Doktorand}{Westfälische Wilhelms-Universität Münster}{Münster, Deutschland}{}
        {Forschungsinteressen:
          High-level Abstraktionen für die parallele Programmierung,
          Nutzung moderner paralleler Prozessoren, insbesondere multi-core CPUs
          und GPUs.}
}
\lang{
\cventry{2005--2010}
        {Diploma degree in computer science with a minor in mathematics}
        {\hspace{5em}(equivalent to a M.\,Sc. degree) University of Münster}{Münster, Germany}{}
        {Final grade in computer science: very good (85 \%)}
}{
\cventry{2005--2010}
        {Diplomabschluss in Informatik mit Nebenfach Mathematik}
        {Westfälische Wilhelms-Universität Münster}{Münster, Deutschland}{}
        {Abschlussnote in Informatik: sehr gut}
}

\lang{
\section{Research Visits}
}{
\section{Forschungsaufenthalte}
}
\lang{
\cventry{02/2014}
        {Visiting researcher (1 Month)}{Edinburgh University}{Edinburgh, UK}{}
        {}
\cventry{07/2013\\--\ 11/2013}
        {Visiting researcher (4 Month)}{Edinburgh University}{Edinburgh, UK}{}
        {Funded by the HiPEAC Network of Excellence}
\cventry{07/2012\\--\ 10/2012}
        {Visiting researcher (3 Month)}{EPPC (Edinburgh Parallel Computing Centre)}{Edinburgh, UK}{}
        {Funded by the HPC-Europa2 project}
}{
\cventry{07.2012\\--\ 10.2012}
        {Forschungsaufenthalt}{Universität Edinburgh}{Edinburgh, Vereinigtes Königreich}{}
        {Gefördert durch das HPC-Europa2 Projekt}
}

\section{Reviewer}
\cvline{}{I have been active as an external reviewer for the
          Parallel Computing Technologies~(PaCT), PARCO, Euro-Par, EuroMPI and CCGrid
          conferences.
          The Science of Computer Programming journal and the
          Journal of Supercomputing.
          As well as the HLPGPU2012 HiPEAC Workshop.
}

\lang{
\section{Languages}
}{
\section{Sprachen}
}
\lang{
\cvlanguage{German}
           {Native}{}
}{
\cvlanguage{Deutsch}
           {Muttersprache}{}
}
\lang{
\cvlanguage{English}
           {Fluent}{}
}{
\cvlanguage{Englisch}
           {fließend}{}
}


\pagebreak

\lang{
\section{Research Projects}
}{
\section{Forschungsprojekte}
}
\lang{
\cventry{since 2011}
        {dOpenCL}{An implementation of the OpenCL standard targeting distributed systems}{}{}
        {Ongoing research. Preliminary results published in: [2,8].}
\cvline{}{Together with other research groups we are leading the development of
          OpenCL implementations targeting distributed systems, by developing
          dOpenCL. dOpenCL allows to program all parallel processors
          (e.g., CPUs or GPUs) of a distributed Systems using OpenCL as the
          single programming model.
          %We plan combining SkelCL and dOpenCL to
          %create a software stack combining the strength of both projects.
          %This will allow for high-level programming of heterogeneous
          %distributed systems.
        }

\cventry{since 2010}
        {SkelCL}{A high-level programming library for heterogeneous systems}{}{}
        {Ongoing research. Preliminary results published in: [1, 3-7, 10-11].}
\cvline{}{I am the lead developer of SkelCL, a novel high-level programming
          model and library for programming heterogeneous systems. It uniquely
          combines algorithmic skeletons, container data types and data
          (re)distribution mechanisms to greatly simplify the programming of
          heterogeneous systems comprising of multiple parallel processors.
          %While raising the level of abstraction we are able to maintain high
          %performance close to manually tuned OpenCL applications and we achieve
          %strong scalability for multi-GPU systems evaluating a real-world
          %application [4].
          %A new algorithmic skeleton for 2D applications makes use of the fast
          %local GPU memory leading to great performance results [6].
          %We are the first to propose a new algorithmic skeleton capturing the
          %important class of allpairs computations, which include applications
          %like n-body simulations or matrix multiplication.
          %By cleverly utilize semantic information of the allpairs computational
          %scheme we are able to achieve high performance competitive to highly
          %tuned BLAS routines [2].
          %Our in-depth application study shows that using our library instead of
          %OpenCL one can save half the lines of code while still achieving 95\%
          %of the performance of a manually tuned implementation [1].
          SkelCL is open source software and can be found online at:
          \url{http://skelcl.uni-muenster.de}.
        }

\cventry{01/2010\\ --\ 09/2010}
        {Diploma project}{Developing a Portable Multi-GPU Skeleton Library}{}{}
        {Results published in [9].}
\cvline{}{As my diploma project I developed the predecessor to SkelCL,
          an innovative high-level programming library for simplified
          programming of GPUs. We published the results of my diploma project in
          [9] and showed, that we can greatly simplify the programming of GPU
          systems without scarifying performance.
          %Especially, we are able to
          %completely eliminate the need for explicit data movement, which is a
          %challenging task in state-of-the-art GPU programming.
          }
}{
}

% \lang{
% \section{Diploma thesis}
% }{
% \section{Diplomarbeit}
% }
% \lang{
% \cvline{Title}
%        {\emph{Developing a Portable Multi-GPU Skeleton Library}}
% }{
% \cvline{Titel}
%        {\emph{Developing a Portable Multi-GPU Skeleton Library}}
% }
% \lang{
% \cvline{Supervisors}
%        {Prof. Sergei Gorlatch, University of Münster}
% }{
% \cvline{Betreuer}
%        {Prof. Sergei Gorlatch, Westfälische Wilhelms-Universität Münster}
% }
% \lang{
% \cvline{Description}
%        {\small While CUDA and OpenCL made general-purpose
%          programming for Graphics Processing Units (GPU) popular,
%          using these programming approaches remains complex and
%          error-prone because they lack high-level abstractions. The
%          especially challenging systems with multiple GPU are not
%          addressed at all by these low-level programming models. We
%          propose SkelCL – a library providing so-called algorithmic
%          skeletons that capture recurring patterns of parallel compu-
%          tation and communication, together with an abstract vector
%          data type and constructs for specifying data distribution. We
%          demonstrate that SkelCL greatly simplifies programming GPU
%          systems. Because the library is implemented using OpenCL, it
%          is portable across GPU hardware of different vendors.}
% }{
% \cvline{Description}
%        {\small While CUDA and OpenCL made general-purpose
%          programming for Graphics Processing Units (GPU) popular,
%          using these programming approaches remains complex and
%          error-prone because they lack high-level abstractions. The
%          especially challenging systems with multiple GPU are not
%          addressed at all by these low-level programming models. We
%          propose SkelCL – a library providing so-called algorithmic
%          skeletons that capture recurring patterns of parallel compu-
%          tation and communication, together with an abstract vector
%          data type and constructs for specifying data distribution. We
%          demonstrate that SkelCL greatly simplifies programming GPU
%          systems. Because the library is implemented using OpenCL, it
%          is portable across GPU hardware of different vendors.}
% }

%\pagebreak

\lang{
\section{Presentations}
}{
\section{Vorträge}
}
\lang{
  \cvline{05/2014}
         {Invited Talk: \emph{SkelCL: High-Level Programming of Multi-GPU
          Systems}\newline \small Institute for Computational and Applied
          Mathematics, University of Münster, Germany}
}{
  \cvline{05.2014}
         {Vorstellung des SkelCL Forschungsprojekts auf dem Workshop on Fast
          Data Processing on GPUs in Dresden, Deutschland.}
}
\lang{
  \cvline{05/2014}
         {Invited Talk: \emph{SkelCL: High-Level Programming of Multi-GPU
          Systems}\newline \small Workshop on Fast Data Processing on GPUs in
          Dresden, Germany.}
}{
  \cvline{05.2014}
         {Vorstellung des SkelCL Forschungsprojekts auf dem Workshop on Fast
          Data Processing on GPUs in Dresden, Deutschland.}
}
\lang{
  \cvline{01/2014}
         {Talk: \emph{Extending the SkelCL Skeleton Library for Stencil
          Computations on Multi-GPU Systems}\newline \small HiStencils 2014
          workshop in Vienna, Austria.}
}{
  \cvline{01.2014}
         {Vorstellung eines Beitrags auf dem HiStencils 2014 Workshop in Wien,
          Österreich.}
}
\lang{
  \cvline{12/2013}
         {Invited Talk: \emph{SkelCL: High-Level Programming of Multi-GPU
          Systems}\newline \small Research group on elementary particle physics,
          University of Wuppertal, Germany.}
}{
  \cvline{12.2013}
         {Vorstellung des SkelCL Forschungsprojekts an der Universität
          Wuppertal, Deutschland.}
}
\lang{
  \cvline{07/2013}
         {Talk: \emph{Introducing and Implementing the Allpairs Skeleton for GPU
          Systems}\newline \small HLPP 2013 workshop in Paris, France.}
}{
  \cvline{07.2013}
         {Vorstellung eines Beitrags auf dem HLPP 2013 Workshop in Paris,
          Frankreich.}
}
\lang{
\cvline{06/2013}
       {Talk: \emph{High-Level Programming for Medical Imaging on Multi-GPU
        Systems using the SkelCL Library}\newline \small ICCS 2013 conference in
        Barcelona, Spain.}
}{
\cvline{06.2013}
       {Vorstellung eines Beitrags auf der ICCS 2013 Konferenz in Barcelona,
        Spanien.}
}
\lang{
\cvline{08/2012}
       {Talk: \emph{Using the SkelCL Library for High-Level GPU Programming of
        2D Applications}\newline \small ParaPhrase 2012 workshop held in
        conjunction with Euro-Par 2012 in Rhodes, Greece.}
%       {Paper Presentation at the Paraphrase workshop (held in conjunction with
%        Euro-Par 2012) in Rhodes, Greece.}
}{
\cvline{08.2012}
       {Vorstellung eines Beitrags auf dem Paraphrase Workshop (durchgeführt
        zusammen mit Euro-Par 2012) in Rhodos, Griechenland.}
}
\lang{
\cvline{06/2012}
       {Talk: \emph{High-Level Programming for Heterogeneous Systems with
        Accelerators}\newline \small PDESoft 2012 workshop in Münster, Germany.}
%       {Presentation of the SkelCL research project at the PDESoft 2012 workshop
%        in M\"unster, Germany.}
}{
\cvline{06.2012}
       {Vorstellung des SkelCL Forschungsprojekts auf dem PDESoft 2012 Workshop
        in Münster, Deutschland.}
}
\lang{
\cvline{05/2012}
       {Talk:\emph{Towards High-Level Programming of Multi-GPU Systems Using
        the SkelCL Library}\newline \small AsHES 2012 workshop held in
        conjunction with IPDPS 2012 in Shanghai, China.}
%       {Paper Presentation at the AsHES 2012 workshop (held in conjunction with
%        IPDPS 2012) in Shanghai, China.}
}{
\cvline{05.2012}
       {Vorstellung eines Beitrages auf dem ASHES 2012 Workshop (durchgeführt
        zusammen mit IPDPS 2012) in Schanghai, China.}
}
\lang{
\cvline{04/2012}
       {Invited talk: \emph{A Skeleton Library for Heterogeneous
        Multi-/Many-Core Systems}\newline \small NAIS workshop in Edinburgh, UK.}
%       {Invited speaker at the NAIS workshop in Edinburgh, UK. Presentation of
%        the SkelCL research project.}
%Presentation of the SkelCL research project
}{
\cvline{04.2012}
       {Vorstellung des SkelCL Forschungsprojekts auf dem NAIS Workshop in
        Edinburgh, Vereinigtes Königreich.}
}
\lang{
\cvline{01/2012}
       {Talk: \emph{Towards a High-Level Approach for Programming Distributed
        Systems with GPUs}\newline \small COST Action IC0805 (``ComplexHPC'')
        meeting in Timisoara, Romania.}
%       {Presentation of the SkelCL research project at COST Action IC0805
%        (``ComplexHPC'') meeting in Timisoara, Romania.}
}{
\cvline{01.2012}
       {Vorstellung des SkelCL Forschungsprojekts beim Treffen der COST Action
         IC0805 (``ComplexHPC'') in Timisoara, Rumänien.}
}
\lang{
\cvline{12/2011}
       {Invited talk: \emph{SkelCL -- A High-Level Programming Library for GPU
        Programming}\newline \small Jülich Supercomputing Centre (JSC), Germany.}
%       {Presentation of the SkelCL research project at the supercomputing center
%        in J\"ulich, Germany.}
}{
\cvline{12.2011}
       {Vorstellung des SkelCL Forschungsprojekts am Supercomputing Centre
        im Forschungszentrum Jülich, Deutschland}
}
\lang{
\cvline{05/2011}
       {Talk: \emph{SkelCL -- A Portable Skeleton Library for High-Level
        GPU Programming}\newline
        \small HIPS 2011 workshop held in conjunction with IPDPS 2011 in
        Anchorange, Alaska, USA.}
%       {Paper presentation at the HIPS 2011 workshop (held in conjunction with
%        IPDPS 2011) in Achorage, USA.}
}{
\cvline{05.2011}
       {Vorstellung eines Beitrages auf dem HIPS 2011 Workshop (durchgeführt
        zusammen mit IPDPS 2011) in Achorage, USA.}
}
\lang{
\cvline{09/2008}
       {Invited talk: \emph{Development of an Online Game as a Student Project}\newline
        \small ITSoftTEAM workshop in Chernihiv, Ukraine.}
%       {Presentation of a student project at the ITSoftTEAM third international
%        project workshop in Chernihiv, Ukraine.}
}{
\cvline{09.2008}
       {Vorstellung eines Studierendenprojekts auf dem dritten internationalen
        ITSoftTeam Projekt Workshop in Chernihiv, Ukraine.}
}


%\pagebreak

\lang{
\section{Publications}
}{
\section{Veröffentlichungen}
}

\lang{
\subsection{Journal Articles}
}{
\subsection{Artikel in Zeitschriften}
}
% Journal Articles
\nociteJn{*}
\bibliographystyleJn{bibcvstyle}
\bibliographyJn{journal}

\lang{
\subsection{Conference Proceedings}
}{
\subsection{Konferenzbeiträge}
}
% Conference Proceedings
\nociteCn{*}
\bibliographystyleCn{bibcvstyle}
\bibliographyCn{conference}

\lang{
\subsection{Book Chapter}
}{
\subsection{Buchkapitel}
}
% Book Chapters
\nociteBk{*}
\bibliographystyleBk{bibcvstyle}
\bibliographyBk{bookchapter}


\section{Teaching}
\cvline{Summer Term 2014}{
        Supervised a student project:
        \emph{Design and implementation of a high-level API for programming heterogeneous clusters}.
      }
\cvline{Winter Term 2013/2014}{
        Supervised a student project:
        \emph{High-level programming of online games in future generation networks}.
      }
\cvline{Summer Term 2013}{
        Lecturer for the course:
        \emph{Introduction to programming with C and C++}.\newline
        Teaching assistant for the course:
        \emph{Multi-core and GPU: Parallel Programming}.
      }
\cvline{Winter Term 2011/2012}{
        Teaching assistant for the course:
        \emph{Operating Systems}.
      }
\cvline{Summer Term 2012}{
        Supervised a student project:
        \emph{High-level programming of heterogeneous parallel systems}.\newline
        Teaching assistant for the course:
        \emph{Multi-core and GPU: Parallel Programming}.
      }
\cvline{Winter Term 2011/2012}{
        Teaching assistant for the seminar:
        \emph{Technical aspects of cloud computing}.\newline
        Teaching assistant for the course:
        \emph{Operating Systems}.
      }
\cvline{Summer Term 2011}{
        Supervised a student project:
        \emph{Internet- and GPU-based Cloud Computing}.\newline
        Teaching assistant for the course:
        \emph{Multi-core and GPU: Parallel Programming}.
      }
\cvline{Winter Term 2010/2011}{
        Supervised a student project:
        \emph{High-level GPU programming}.
      }

\section{Supervised Students\newline
         \footnotesize(All students are co-supervised with Prof. Sergei Gorlatch)}
\subsection{Currently active}
\cvline{since 04/2013}{Bachelor thesis of André Lüers:
                 \emph{Evaluation of the Skeleton Library FastFlow}}
\cvline{since 04/2014}{Bachelor thesis of Lars Klein:
                 \emph{A Parallel Implementation of the T-CUP Software using the
                       SkelCL Library}}
\cvline{since 05/2013}{Bachelor thesis of Fabian Hall:
                 \emph{Design and implementation of an OpenCL compatibility API
                       for the SkelCL Library}}
\subsection{Completed}
\cvline{01/2014}{Master thesis of Michael Olejnik:
                       \emph{A GPU-based Classification Framework for
                             HIV Resistance Prediction}
                       \footnotesize (co-supervised with Dr. habil. Dominik Heider)}
\cvline{01/2014}{Master thesis of Stefan Breuer:
                       \emph{Extending the SkelCL Library for 
                             Stencil Computations}}
\cvline{11/2013}{Diploma thesis of Wadim Hamm:
                       \emph{Development of a Divide \& Conquer Skeleton for
                             SkelCL}}
\cvline{07/2013}{Bachelor thesis of Matthias Droste:
                       \emph{Evaluation of the Skeleton Library SkePU}}
\cvline{06/2013}{Bachelor thesis of Kai Kientopf:
                       \emph{Implementation of the Needleman-Wunsch Algorithm
                             and the Breath-First-Search with the SkelCL Library}}
\cvline{06/2013}{Master thesis of Florian Quinkert:
                       \emph{A Model for Predicting Work Distribution in
                             Heterogeneous Systems and its Implementation
                             in the SkelCL Library}}
\cvline{03/2013}{Master thesis of Malte Friese:
                 \emph{Extending the Skeleton Library SkelCL with a Skeleton for
                       Allpairs Computations}}
\cvline{03/2013}{Bachelor thesis of Sebastian Mißbach:
                 \emph{Implementing the LU-Decomposition and the Mersenne-Twister
                       with the SkelCL Library}}
\cvline{03/2013}{Bachelor thesis of Patrick Schiffler:
                 \emph{Performance Analysis of SkelCL using B+ Tree Traversal and
                       3D Jacobi Stencil Computation}}
\cvline{01/2013}{Diploma thesis of Markus Blank-Burian:
                 \emph{Simulation and Analysis of Two-Dimensional Turbulences on
                       Parallel Computer Architectures}
                 \footnotesize (co-supervised with Prof. Gernot Münster)}
\cvline{06/2012}{Diploma thesis of Matthias Buß:
                 \emph{Adding Multidimensional Data Types to the Multi-GPU
                       Skeleton Library SkelCL}}
\cvline{09/2011}{Bachelor thesis of Michael Olejnik:
                 \emph{Investigating the Use of GPUs for Radix Sort}}
\cvline{09/2011}{Bachelor thesis of Jan Gerd Tenberge:
                 \emph{Extending the SkelCL Library with Iterators}}
\cvline{08/2011}{Bachelor thesis of Stefan Breuer:
                 \emph{Enhancing SkelCL's MapOverlap Skeleton}}
\cvline{08/2011}{Bachelor thesis of Tobias Günnewig:
                 \emph{Developing a Library for Manipulating Source Code of
                   C-based Languages}
                 \footnotesize (co-supervised with Philipp Kegel)}
% \lang{
% \section{Computer Skills}
% }{
% \section{Computer Fähigkeiten}
% }
% \lang{
% \cvline{Some Experience}
%        {Databases, Machine Learning, Java, Ruby, Python, Haskell, Erlang}
% }{
% \cvline{Einige Erfahrungen}
%        {Datenbanken, Maschinelles Lernen, Java, Ruby, Python, Haskell, Erlang}
% }
% \lang{
% \cvline{Extended Experience}
%        {C/C++ Programming, GPU Programming (OpenCL, CUDA), Parallel Programming,
%         Unix/Linux}
% }{
% \cvline{Erweiterte Erfahrungen}
%        {C/C++ Programmierung, GPU Programmierung (OpenCL, CUDA),
%         Parallele Programmierung, Unix/Linux}
% }

\end{document}


%% end of file `template_en.tex'.
